\documentclass[12pt]{article}
\usepackage[brazil]{babel}
\usepackage[a4paper, total={6.5in, 9.5in}]{geometry}
\usepackage[utf8]{inputenc}

\usepackage{authblk}
\usepackage{lipsum}
\usepackage{xcolor}
\usepackage{listings}
\usepackage[normalem]{ulem}
\usepackage{amssymb}
\usepackage{float}
\usepackage{graphicx}
\usepackage{amsmath}
\usepackage{amsthm}
\usepackage{amssymb}
\newtheorem{theorem}{Theorem}[section]
\newtheorem{lemma}[theorem]{Lemma}

\definecolor{codegreen}{rgb}{0,0.6,0}
\definecolor{codegray}{rgb}{0.5,0.5,0.5}
\definecolor{codered}{rgb}{0.8,0,0}
\definecolor{backcolour}{rgb}{0.95,0.95,0.92}

\usepackage{inconsolata}
\lstset{
    language=sh,
    backgroundcolor=\color{backcolour},   
    commentstyle=\color{codegreen},
    keywordstyle=\color{blue},
    numberstyle=\tiny\color{codegray},
    stringstyle=\color{codered},
    basicstyle=\ttfamily\small,
    numberstyle=\footnotesize,
    numbers=left,
    backgroundcolor=\color{white},
    frame=single,
    tabsize=2,
    rulecolor=\color{white},
    title=\lstname,
    escapeinside={\%*}{*)},
    breaklines=true,
    breakatwhitespace=true,
    framextopmargin=2pt,
    framexbottommargin=2pt,
    inputencoding=utf8,
    extendedchars=true,
    showstringspaces=false,
    literate={á}{{\'a}}1 {ã}{{\~a}}1 {é}{{\'e}}1 {Ó}{{\'O}}1 {Ã}{{\~A}}1 {í}{{\'i}}1 {ó}{{\'o}}1 {ç}{{\.c}}1 {ê}{{\^e}}1,
}

% \pagecolor[rgb]{0.1,0.1,0.1} %black
% \color[rgb]{0.75,0.75,0.75} %grey

\title{\textbf{Segurança Computacional\\ \Large{Cifra de Vignère}}}
\author{Pedro Henrique de Brito Agnes, 18/0026305 \vspace{-2ex}}
\affil{Dep. Ciência da Computação - Universidade de Brasília (UnB) \vspace{-2ex}}
\date{}

\begin{document}
\maketitle

\section{Cifrador/Decifrador}
Para a implementação do cifrador e decifrador da cifra de Vigenère foi criado um programa em python na versão 3.6, que está no arquivo \texttt{src/parte1.py}. Na execução, o programa aceita alguns parâmetros, porém pode ser executado sem nenhum da seguinte forma:

\begin{lstlisting}
python src/parte1.py        # cifrador
python src/parte1.py -d     # decifrador
\end{lstlisting}

Pela abordagem acima, será solicitada durante a execução, uma mensagem/criptograma e uma chave e terá como saída a mensagem cifrada/decifrada de acordo com a cifra de Vigenère e com a chave informada. Conforme informado, o programa aceita certos parâmetros. Para visualizar a lista completa e o que fazem, pode ser utilizada a flag \texttt{-h} ou \texttt{--help}:

\begin{lstlisting}
python src/parte1.py -h
\end{lstlisting}

O primeiro parâmetro que pode ser passado é um arquivo que conterá a mensagem seguido do segundo que também é um arquivo onde será passada a chave. Opcionalmente, pode-se passar um arquivo de saída com a flag \texttt{-o} e, caso não seja passada, a saída (mensagem cifrada/decifrada) será printada no terminal.

Para cifrar uma mensagem, pode ser executado o comando abaixo como exemplo:

\begin{lstlisting}
python src/parte1.py sample/msg.txt sample/key.txt -o sample/out.txt
\end{lstlisting}

Onde \texttt{sample/msg.txt} é a mensagem de exemplo no diretório \texttt{sample}, \texttt{sample/key.txt} é a chave e \texttt{sample/out.txt} é o arquivo onde será impressa a saída. Da mesma forma, para decifrar a mensagem cifrada com o comando anterior, pode ser usado o comando a seguir:

\begin{lstlisting}
python src/parte1.py sample/out.txt sample/key.txt -o sample/decifrada.txt -d
\end{lstlisting}

A flag \texttt{-d} ao final é que indica ao programa que a mensagem será decifrada e não cifrada. No exemplo, a saída foi definida para aparecer no arquivo \texttt{sample/decifrada.txt}.

\subsection{Aspectos Técnicos}
Toda a lógica utilizada para a implementação do cifrador/decifrador estão em um arquivo separado \texttt{src/vigenere/cipher.py}. Este arquivo tem uma função principal \texttt{cifra} que recebe uma mensagem, uma chave e uma variável booleana \texttt{opt}, que define se a mensagem deve ser cifrada (\texttt{false}) ou decifrada (\texttt{true}). Para as operações, é basicamente feita uma soma dos valores ascii da letra atual da mensagem subtraída do ascii da letra 'a' com o ascii da letra atual da \textit{keystream}.
Com o valor obtido na operação anterior, é realizado o módulo por 26 (quantidade de letras no alfabeto) e assim, é somado novamente o valor do ascii de 'a' para obter o caractere do criptograma. Este processo é feito em loop até que se chegue ao final da mensagem, porém caracteres que não sejam letras ou acentuados não são processados e se mantém inalterados no output. Já a decifração funciona de forma muito similar à citada acima, mudando apenas a operação de soma para subtração entre o ascii da letra da mensagem e da \textit{keystream}.

\section{Ataque}
Para o ataque à cifra de Vigenère foi criado o programa em python 3.6 ou superior que se encontra no arquivo \texttt{src/parte2.py}. O programa aceita certos parâmetros, mas pode ser executado sem nenhum da seguinte forma:

\begin{lstlisting}
python src/parte2.py        # inglês
python src/parte2.py -p     # português
\end{lstlisting}

Com a chamada acima, será requisitado um criptograma pelo input do terminal e terá como output a chave encontrada.

Para acessar a lista de parâmetros e o que fazem, existe a flag \texttt{-h} ou \texttt{--help}:

\begin{lstlisting}
python src/parte2.py -h
\end{lstlisting}

O primeiro parâmetro que pode ser passado é o arquivo em que o criptograma se encontra, e pode ser definido um arquivo para o output por meio da flag \texttt{-o}. O comando abaixo vai realizar a tentativa de descobrir a chave do criptograma encontrado no arquivo \texttt{sample/out.txt} considerando que a linguagem da mensagem original era o inglês e vai produzir esta chave no arquivo \texttt{sample/crackedkey.txt}:

\begin{lstlisting}
python src/parte2.py sample/out.txt -o sample/crackedkey.txt
\end{lstlisting}

Da mesma forma, para decifrar a mensagem em português do arquivo \texttt{sample/out\_pt.txt}, pode-se usar o comando abaixo, onde a flag \texttt{-p} indica que a mensagem é em português:

\begin{lstlisting}
python src/parte2.py sample/out_pt.txt -o sample/crackedkey.txt -p
\end{lstlisting}

\subsection{Aspectos Técnicos}
A lógica do ataque foi toda incluída no arquivo \texttt{src/vigenere/attack.py}. Para a decifração da chave, foi utilizada a análise de frequências das letras. Para isso, foi obtida a lista de frequências padrão de cada letra na língua inglesa e portuguesa, e elas são usadas no programa para obter a diferença da frequência das letras na mensagem. A ocorrência que tiver a menor diferença representa uma letra da chave decifrada e este processo é feito em loop até encontrar toda a chave de acordo com seu tamanho obtido.

Para obter o tamanho da chave, primeiramente são obtidas sequências de três letras do criptograma que se repetem e é analisada a distância entre elas. Os múltiplos dessa distância são "marcados" a cada vez e o número que tiver a maior quantidade de marcas ao final é o tamanho da chave. Esta abordagem funciona para muitos casos se o tamanho da chave for um número primo. Caso contrário, o valor retornado será algum dos múltiplos do tamanho da chave.

Em alguns casos, pode ser que o algoritmo de quebra da cifra de Vigenère não funcione corretamente. Um exemplo seria com mensagens muito pequenas já que é feita a análise de frequências. O tamanho mínimo testado que gerou o resultado esperado foi para uma mensagem de 1500 letras.

\end{document}